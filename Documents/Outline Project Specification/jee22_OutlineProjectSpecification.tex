\documentclass[11pt,fleqn,twoside]{article}
\usepackage{makeidx}
\makeindex
\usepackage{palatino} %or {times} etc
\usepackage{plain} %bibliography style 
\usepackage{amsmath} %math fonts - just in case
\usepackage{amsfonts} %math fonts
\usepackage{amssymb} %math fonts
\usepackage{lastpage} %for footer page numbers
\usepackage{fancyhdr} %header and footer package
\usepackage{mmpv2} 
\usepackage{url}

% the following packages are used for citations - You only need to include one. 
%
% Use the cite package if you are using the numeric style (e.g. IEEEannot). 
% Use the natbib package if you are using the author-date style (e.g. authordate2annot). 
% Only use one of these and comment out the other one. 
\usepackage{cite}
%\usepackage{natbib}

\begin{document}

\name{James Edward Euesden}
\userid{jee22}
\projecttitle{A Toolkit For Reporting On Metagenome Assembly Quality}
\projecttitlememoir{A toolkit for reporting on metagenome assembly quality} %same as the project title or abridged version for page header
\reporttitle{Outline Project Specification}
\version{1.0} % Correct to 1.0 when in Release
\docstatus{Release} % or Release
\modulecode{CS39440}
\degreeschemecode{G401}
\degreeschemename{BSC Computer Science (inc Integrated Industrial and Professional Training)}
\supervisor{Amanda Clare} % e.g. Neil Taylor
\supervisorid{afc}
\wordcount{}

%optional - comment out next line to use current date for the document
%\documentdate{10th February 2014} 
\mmp

\setcounter{tocdepth}{3} %set required number of level in table of contents

%==============================================================================
\section{Project description}
%==============================================================================
The project aims to produce a software application to serve as a toolkit for reporting on metagenome assembly quality. This toolkit should allow a user to provide input of a metagenome\cite{metagenome} (most likely in the FASTA format\cite{fasta}) and receive output of a report detailing the quality of their metagenome. This project will involde research into metagenomics and the designing and developing of an application to carry out the task.

Before any technical work can be done, I must determine what quality is in metagenomics. Quality could be multiple things, and could be up for interpretation depending what result the user wishes to see. The user may ask questions such as: Is the assembly found in nature, or is it  a chimera made by the assembly? Is the assembly long enough to be useful, or is it so long that it may be irregular? Understanding the users requirement on quality will help me develop the application with the focus on the user.

It could also be useful to a user if the report suggests places where the metagenome may be split if there are instances where it seems likely that the assembly doesn't seem right e.g. reads combined where they should not have been. We can never truly say "This is good quality", as we don't know what we have to begin with, but the aim is to present a report on what areas of the assembly could be looked at in closer detail and why. There is also the potential for using a large database of known genomes to see if any known genome matches all or part of the provided assembly.

Quality control in metagenomics is huge. Without good quality control, a user may attempt to synthesize a gene based on an assembly that has no way of existing in nature, and not achieve the result they were originally aiming for. While the application may not be able to give exact results on how `good' an assembly is, by highlighting significant areas where there may be issues, it will potentially aid the field of metagenomics in understanding the samples taken and reaching assemblies in better ways.

%==============================================================================
\section{Proposed tasks}
%==============================================================================
\textbf{Research metagenomics} - Understand what they are, how are assemblies created, and how do they look in text-form? I will need to have basic understanding of the domain before I can begin technical work. Reading papers on metagenomics and current processes of determining quality will help me do this.
\newline
\noindent
\textbf{Investigate existing/similar tools} - There are a number of tools and techniques that exist for quality control checks for genomes. Some of the techniques I intend to investigate are K-mer\cite{kmer} counting, GC counting\cite{gccount}, n50 and anything else I discover. Applications such as Jellyfish\cite{jellyfish}, BFC\cite{bfc}, REAPR\cite{reapr} and QC Chain\cite{qcchain} already do some of these techniques. Looking at these tools may help me understand the domain better and how my final technical product should behave. This investigation will also involve finding good ways of visually representing the assembly in the quality report.
\newline
\noindent
\textbf{Set up local environment and version control} - Initially, I plan to use the Java language to develop my application, using the IntelliJ IDE. For version control, I will use git and keep my repository on my personal GitHub account during the projects lifetime. During this period, I will also decide which way to develop the application, whether I take a plan driven approach or an agile approach, using all or elements from Scrum, XP and Feature Driven Development.
\newline
\noindent
\textbf{Development} - To begin with, I will just program some basics, such as GC content counting, both to learn and to start the application, and approach each technique as an addition to the application, building on it iteratively to quickly have a minium viable product that can easily be expanded upon and maintained over time. 
\newline
\noindent
\textbf{Implement more advanced techniques} - Once a minimum product has been made, I intend to further develop the application to provide a detailed visual report about the quality of the metagenome provided. The report will give the user enough detailed information that they can use in order to find why the application gives the results it does, and potentially offer solutions as to what they might find useful to look at either to improve the metagenome, or discover where the irregularities are located. This will involve more advanced quality control checks, and the potential for checking databases for similar/matches to existing known genome sequences through using BLAST\cite{blast}.
\newline
\noindent
\textbf{Construct a set of test scripts and files} - In order to determine whether my application is successful in its tasks, I will write a number of test scripts for its functionality, and use real and artificially created data where I have expected and assumed outputs and see if my application gives me the anticipated results.
\newline
\noindent
\textbf{Compare outputs/the report} - Compare the results from my program with outputs from similar applications that do similar tasks, or communicate with those in the field to determine whether the application has some degree of success with their expectations.
\newline
\noindent
\textbf{Project meetings and online blog} - Attend project meetings with my supervisor (minimum) once a week, and discuss my progress and plans. These will also be documented on my blog, and will reflect the stories I am taking into each week to work on.
\newline
\noindent
\textbf{Preparation for demonstrations} - There are two demonstrations, a mid-project demonstration in the week before Easter and a final demonstration after the submission of my technical work and final report. Both of these demonstrations will be planned for before being given, and through them I hope to show my markers the function of my application, any research I have conducted and any technical challenges or interesting sections of my application. 

%==============================================================================
\section{Project deliverables}
%==============================================================================
\textbf{Mid project demonstration notes} - A compiled set of notes used in planning and giving a mid-project demonstration. This will be included in the appendix of the final report.
\newline
\noindent
\textbf{Test Files} - Taken from sources I am able to use, or artificially generated for checking that my application does indeed return a report of expected quality when used in practice. These will be as part of the technical submission.
\newline
\noindent
\textbf{Test Scripts} - Most likely using JUnit as the base, these will be included in the technical submission of the software application, and where relevent run with the test files provided. A series of scripts to test the expected functionality of my application, included in the technical submission.
\newline
\noindent
\textbf{Software Application} - Takes the input of a metagenome assembly, or of output results from other software results, and returns a report on the quality of the metagenome in question. This will be the bulk of the technical work and the focus of this project.
\newline
\noindent
\textbf{Story cards and planning documents} - Within the final report appendix will be a document detailing the stories I undertook during the project process and any planning documents I created to aid the project development.
\newline
\noindent
\textbf{Fiinal Report} -  The final report documents my process over the projects life time, the work done and acknowledgements to any papers and journals read during my research, third party software and tools used during the project. An appendix will be attached, including any deliverables that are required to be included but not part of the report itself.
\newline
\noindent
\textbf{Final Demonstration} - While there will be no documents, this is an event that will take place and I will need to consider how I structure the demonstration and how to present it during the time of my project
%
% Start to comment out / remove the following lines. They are only provided for instruction for this example template.  You don't need the following section title, because it will be added as part of the bibliography section. 
%
%==============================================================================
%\section*{Your Bibliography - REMOVE this title and text for final version}
%==============================================================================
%
%You need to include an annotated bibliography. This should list all relevant web pages, books, journals etc. that you have consulted in researching your project. Each reference should %include an annotation. 

%The purpose of the section is to understand what sources you are looking at.  A correctly formatted list of items and annotations is sufficient. You might go further and make use of %bibliographic tools, e.g. BibTeX in a LaTeX document, could be used to provide citations, for example \cite{NumericalRecipes} \cite{MarksPaper} \cite[99-101]{FailBlog} \cite{kittenpic_ref}.  %The bibliographic tools are not a requirement, but you are welcome to use them.   

%You can remove the above {\em Your Bibliography} section heading because it will be added in by the renewcommand which is part of the bibliography. The correct annotated bibliography %information is provided below. 
%
% End of comment out / remove the lines. They are only provided for instruction for this example template. 
%

\nocite{*} % include everything from the bibliography, irrespective of whether it has been referenced.

% the following line is included so that the bibliography is also shown in the table of contents. There is the possibility that this is added to the previous page for the bibliography. To address this, a newline is added so that it appears on the first page for the bibliography. 
%\newpage
\addcontentsline{toc}{section}{Initial Annotated Bibliography} 

%
% example of including an annotated bibliography. The current style is an author date one. If you want to change, comment out the line and uncomment the subsequent line. You should also modify the packages included at the top (see the notes earlier in the file) and then trash your aux files and re-run. 
%\bibliographystyle{authordate2annot}
\bibliographystyle{IEEEannot}
\renewcommand{\refname}{Annotated Bibliography}  % if you put text into the final {} on this line, you will get an extra title, e.g. References. This isn't necessary for the outline project specification. 
\bibliography{mmp} % References file

\end{document}
