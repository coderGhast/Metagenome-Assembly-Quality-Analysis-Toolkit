\chapter{Third-Party Code and Libraries}
%If you have made use of any third party code or software libraries, i.e. any code that you have not designed and written yourself, then you must include this appendix. 

%As has been said in lectures, it is acceptable and likely that you will make use of third-party code and software libraries. The key requirement is that we understand what is your original work and what work is based on that of other people. 

%Therefore, you need to clearly state what you have used and where the original material can be found. Also, if you have made any changes to the original versions, you must explain what you have changed. 

%As an example, you might include a definition such as: 

%Apache POI library � The project has been used to read and write Microsoft Excel files (XLS) as part of the interaction with the client�s existing system for processing data. Version 3.10-FINAL was used. The library is open source and it is available from the Apache Software Foundation 
%\cite{apache_poi}. The library is released using the Apache License 
%\cite{apache_license}. This library was used without modification. 

jQuery - The jQuery library has been used to display tabs for the results page and help detecting user clicks on HTML5 Canvas elements. The library is open source and it is available from the jQuery Foundation.
\cite{jquery}. The library is released using the MIT license.
\cite{mitlicense}. This library was used without modification.
\newline
Plotly.js - A library for displaying graphs and charts, used in the application for creating and displaying the GC content results. The library is open source and is available from Plotly.
\cite{plotly}. The library is released under the MIT license.
\cite{mitlicense}. This library was used without modification.
\newline
Spring Framework - The project is built using Java and Spring. The Spring Framework supports the application for being hosted as a web service. The library is open source and it is available from Pivotal Software.
\cite{spring}. The library is released under the Apache License 2.0.
\cite{apachelicense}. This library was used without modification.
\newline
Thmyeleaf - The project used Thymeleaf for accessing data from the Java code with the Spring Framework Model and display it in the View through JavaScript. The library is open source and it is available from The Thymeleaf Team.
\cite{thymeleaf}. The library is released under the Apache License 2.0.
\cite{apachelicense}. This library was used without modification.

\section{Tabs for results}
The code for displaying the results in tabs using CSS3 and Jquery\cite{tabcode}
\subsection{HTML}
\begin{verbatim}
<ul id="tabs">
    <li><a href="#" name="tab1">One</a></li>
    <li><a href="#" name="tab2">Two</a></li>
    <li><a href="#" name="tab3">Three</a></li>
    <li><a href="#" name="tab4">Four</a></li>    
</ul>

<div id="content"> 
    <div id="tab1">...</div>
    <div id="tab2">...</div>
    <div id="tab3">...</div>
    <div id="tab4">...</div>
</div>
\end{verbatim}
\subsection{CSS}
\begin{verbatim}
#tabs {
  overflow: hidden;
  width: 100%;
  margin: 0;
  padding: 0;
  list-style: none;
}

#tabs li {
  float: left;
  margin: 0 .5em 0 0;
}

#tabs a {
  position: relative;
  background: #ddd;
  background-image: linear-gradient(to bottom, #fff, #ddd);  
  padding: .7em 3.5em;
  float: left;
  text-decoration: none;
  color: #444;
  text-shadow: 0 1px 0 rgba(255,255,255,.8);
  border-radius: 5px 0 0 0;
  box-shadow: 0 2px 2px rgba(0,0,0,.4);
}

#tabs a:hover,
#tabs a:hover::after,
#tabs a:focus,
#tabs a:focus::after {
  background: #fff;
}

#tabs a:focus {
  outline: 0;
}

#tabs a::after {
  content:'';
  position:absolute;
  z-index: 1;
  top: 0;
  right: -.5em;  
  bottom: 0;
  width: 1em;
  background: #ddd;
  background-image: linear-gradient(to bottom, #fff, #ddd);  
  box-shadow: 2px 2px 2px rgba(0,0,0,.4);
  transform: skew(10deg);
  border-radius: 0 5px 0 0;  
}

#tabs #current a,
#tabs #current a::after {
  background: #fff;
  z-index: 3;
}

#content {
  background: #fff;
  padding: 2em;
  height: 220px;
  position: relative;
  z-index: 2; 
  border-radius: 0 5px 5px 5px;
  box-shadow: 0 -2px 3px -2px rgba(0, 0, 0, .5);
}
\end{verbatim}
\subsection{Jquery}
\begin{verbatim}
<script src="http://code.jquery.com/jquery-1.7.2.min.js"></script>
<script>
$(document).ready(function() {
    $("#content").find("[id^='tab']").hide(); // Hide all content
    $("#tabs li:first").attr("id","current"); // Activate the first tab
    $("#content #tab1").fadeIn(); // Show first tab's content
    
    $('#tabs a').click(function(e) {
        e.preventDefault();
        if ($(this).closest("li").attr("id") == "current"){ //detection for current tab
         return;       
        }
        else{             
          $("#content").find("[id^='tab']").hide(); // Hide all content
          $("#tabs li").attr("id",""); //Reset id's
          $(this).parent().attr("id","current"); // Activate this
          $('#' + $(this).attr('name')).fadeIn(); // Show content for the current tab
        }
    });
});
</script>
\end{verbatim}