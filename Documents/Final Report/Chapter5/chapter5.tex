\chapter{Evaluation}

\section{Requirements Analysis}
How well did I assess the requirements of the project? How did this tie into my Project objectives? Did it impact my design? What about selecting to do GC content over k-mer frequency analysis? Paralysis of choice?

\section{Technical Achievement}
Do I think I did something technical? What did I actually learn? Was it useful to learn? Could i have done something better? Did the software meet the requirements set out? was it of good quality? Could the design have been done differently, or is it good? What about my technology decisions?

\subsection{Future Work}
If I were to continue developing this project, what would I begin with, and how would I go from there? 
\subsubsection{Improvements}
What is in the code currently that could be improved? Refactoring? Functionality? Validation
Improvements on the Superframe? Displaying GC content could be done with my own chart rather than with Plotly. Better way of reporting on when GC Content is drastically different. Window sizes that change and slide.
\subsubsection{Additional Functionality}
What do I feel I would like to implement? k-mer frequency analysis? Tie in to BLAST? File submission? More tests on larger systems? Floor is open from there to find other ways of determining quality.

\section{Project Management}
What was my process like? Did I find it helped or hindered? Could I have tried something different or been better at it? What was good about it?

\section{Final Conclusion}
Overall, how do I think the project did?
If I did start the project again, what would I do different (technical, process, etc? )
Did I learn things, domain and technical?

% Examiners expect to find in your dissertation a section addressing such questions as:

%\begin{itemize}
%   \item Were the requirements correctly identified? 
%   \item Were the design decisions correct?
%   \item Could a more suitable set of tools have been chosen?
%   \item How well did the software meet the needs of those who were expecting to use it?
%   \item How well were any other project aims achieved?
%   \item If you were starting again, what would you do differently?
%\end{itemize}

% Such material is regarded as an important part of the dissertation; it should demonstrate that you are capable not only of carrying out a piece of work but also of thinking critically about how you did it and how you might have done it better. This is seen as an important part of an honours degree. 

% There will be good things and room for improvement with any project. As you write this section, identify and discuss the parts of the work that went well and also consider ways in which the work could be improved. 

% Review the discussion on the Evaluation section from the lectures. A recording is available on Blackboard. 
