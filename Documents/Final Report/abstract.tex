\thispagestyle{empty}

\begin{center}
    {\LARGE\bf Abstract}
\end{center}

Metagenomics is the study of environmental samples of genomic data where the contents of the data are potentially unknown and unclear. There is a need for a quality analysis toolkit in order to help researchers avoid analyzing and interpreting bad data, leading to  wasting time and money.

The project has designed and produced a prototype toolkit (Khimeta) that aims to meet this need and report on the quality of metagenome assembly data. It provides a user with break downs of the contiguous reads within their data, the GC content of windows of each contiguous read, what Open Reading Frames can be found in them and comparisons between areas of high GC content and Open Reading Frame locations.

It has been tested on artificial and real metagenomic data to determine the usefulness of the prototype, and a number of areas of additional functionality have been identified for future development. This report serves to discuss the analysis, design, implementation, testing and evaluation of the project.
