\chapter{Definitions of Technical Terms}
\textbf{DNA sequence:}
A sequence of letters that indicate the order of nucleotides within DNA (ATGC). It has the potential to represent information which has direct impact on the functions of living things.
\newline
\textbf{Read:}
A short subsequence of DNA after a DNA sample has been run through a Sequencer. Often between 50-100 characters of DNA base pairs long.
\newline
\textbf{Sequencer}
Breaks down and reads a DNA sample to produce many small reads in text format. Attempts to determine the order of the four bases, ATGC. The text character `N' may also be included, where a sequencer encounters an error, due to sequencing technology limitations and/or difficulties reading DNA structures.
\newline
\textbf{Bases}
G (guanine), C (cytosine), A (adenine) and T (thymine). Combinations of these bases create the DNA sequences.
\newline
\textbf{Genome:}
A sequence that fully represents the genes or genetic material within a cell or organism.
\newline
\textbf{Assembler:}
An assembler is used to try and align and merge reads to try and create contiguous reads that represent the original DNA sequence that the reads have been created from by a DNA sequencer.
\newline
\textbf{Contig:}
A contiguous sequence that represents a region of DNA, made up of overlapping reads from a DNA sequencer. Produced by assembly software.
\newline
\textbf{Codon}
A series of three DNA nucleotides that represent an amino acid or stop signal, e.g. GAC, CTA, or stop signals such as TGA, TAG, TAA.