\chapter{Testing}

% Detailed descriptions of every test case are definitely not what is required here. What is important is to show that you adopted a sensible strategy that was, in principle, capable of testing the system adequately even if you did not have the time to test the system fully.

% Have you tested your system on �real users�? For example, if your system is supposed to solve a problem for a business, then it would be appropriate to present your approach to involve the users in the testing process and to record the results that you obtained. Depending on the level of detail, it is likely that you would put any detailed results in an appendix.

% The following sections indicate some areas you might include. Other sections may be more appropriate to your project. 

\section{Overall Approach to Testing}
Using TDD and refactoring, only did as much development as was required to make tests pass based upon my requriements for each bit of functionality. Also developed test files, both completely artifical to see expected output results, and using real assembly data that I then combined and modified together to result in artifical chimeras to see if I could detect them by looking at the output of my charts.

\section{Automated Testing}
\subsection{Unit Tests}
Used a number of unit tests, writing tests before writing any functional code. Did not write tests for setters and getters, but wrote tests for expected functionality. Made test cases for things such as the ORF Finding (show screenshot of test and tests passing).

\subsection{Integration Testing}
Testing of reading in files. Just asserting the data is as expected, and what happens with blank files, or erronous data files. Does the system handle it gracefully? (Show examples of those test files and what happens with bad files)

\subsection{User Interface Testing}
Tested to see that it worked using Chrome's Developer Tools utility, testing in Chrome, Firefox and Microsoft Edge. (screenshots of how each page looks for each, including results and charts)
Testing via a test table (Provided test table) - Tested files with known content and expected results to see response. Tested by viewing what I expected to happen, backing this up with the test table e.g. Click to view information, expected graph results based on artifical test files

\subsection{Stress Test}
Set goal of dealing with large files. Tested with large files to see if the system could process them on my laptop. Could be expanded to deal with much larger files on a system that could handle larger data files with more memory. For my laptop though, dealt with the files fine, even if some took some time to process due to my laptops processing speed.
Pasting large amounts of data in a single contig file - Result and time to process file and then inspection time
Pasting large amounts of data spread across multiple contigs - Result and time to process file and then inspection time
